\documentclass[12pt,a4paper]{ctexart}
\usepackage{amsmath}
\usepackage{amssymb}
\usepackage{geometry}
\usepackage{graphicx}
\usepackage{enumitem}
\usepackage{fancyhdr}
\usepackage{lastpage}
\usepackage{xeCJK}
\usepackage{unicode-math}

% 设置中文字体
\setCJKmainfont{STSong}[BoldFont=STHeiti]
\setCJKsansfont{STHeiti}
\setCJKmonofont{STFangsong}

% 设置页面边距
\geometry{top=2.54cm, bottom=2.54cm, left=3.18cm, right=3.18cm}

% 设置页眉页脚
\pagestyle{fancy}
\fancyhf{}
\renewcommand{\headrulewidth}{0.4pt}
\renewcommand{\footrulewidth}{0.4pt}
\fancyhead[C]{高中数学试卷}
\fancyfoot[C]{第 \thepage 页 / 共 \pageref{LastPage} 页}

% 设置数学公式
\allowdisplaybreaks
\setlength{\parindent}{2em}
\setlength{\parskip}{0.5em}
\linespread{1.5}

% 设置标题和页眉的高度
\setlength{\headheight}{15pt}

\begin{document}

\begin{center}
    \Large\textbf{高中数学试卷}
\end{center}

\vspace{1em}

绝密 文启 用前
2024 年普 通高 等学 校招 生全 国统 一考 试 \\$\left( 新课 标 I 卷 》

数学
本试 卷共 10 页 ,19 小题 , 满分 150 分 .
注意 事项 :

\#\#\# 1. 答题 前 , 先将 自己 的姓 名 、 准考 证号 、 考场 号 、 座位 号填 写在 试卷 和答 题卡 上 , 并将 准考 证号 条形 码粘 贴在 答题 卡上 的指 定位 置 .

\#\#\# 2. 选择 题的 作答 : 每小 题选 出答 案后 , 用 28 铅笔 把答 题卡 上对 应题 目的 答案 标号 涂黑 . 写在 试卷 、 草稿 纸和 答题 卡上 的非 答题 区域 均无 效 .

\#\#\# 3. 填空 题和 解答 题的 作答 : 用黑 色签 字笔 直接 答在 答题 卡上 对应 的答 题区 域内 . 写在 试卷 、 草稿 纸和 答题 卡上 的非 答题 区域 均无 效 .

\#\#\# 4. 考试 结束 后 , 请将 本试 卷和 答题 卡一 并上 交 .
一 、 选择 题 : 本题 共 8 小题 , 每小 题 5 分 , 共 40 分 . 在每 小题 给出 的四 个选 项中 , 只有
一个 选项 是正 确的 . 请把 正确 的选 项填 涂在 答题 卡相 应的 位置 上 .

3 =
w-S<x° <5\},B=\$\\left\\\\$\\{-3,-1,0,23\right)\$ l AN B= (


\\\#\\\#\\\# 1. 己夕…口仝茎签\\\_"4二\\\{ )


- A. \\\{-1,0\\right\\\\\\\}\\$\$ 
- B. 「2,3」 
- C. \\$\left\\$\\\\\{-3,-1,0\\right\\\\\\\}\\$\$ 
- D. \\$\left\\$\\\\\{-1,0,2\\right\\\\\\\}\\$\$

2 Babi 则 z= (  )


- A. -1-i 
- B. -1+i 
- C. 1-i 
- D. 1+i

\\\#\\\#\\\# 3. 已知 向量 5= (0,1),5=(2,x), 若 L \\$\left( -45\right)\$, 则 x= ( )

- A. -2 
- B. -1 
- C. 1 
- D. 2

\\\#\\\#\\\# 4. 已知 cos\\$\left(X+ 厂 \right)\$=m,tanQ tan 户 =2 , 则 cos\\$\left(X- 厂 \right)\$ = ( )

- A. -3m 
- B. -2 c 不 
- D. 3m
3 3

\\\#\\\#\\\# 5. 已知 圆柱 和圆 锥的 底面 半径 相等 , 侧面 积相 等 , 且它 们的 高均 为 /3 , 则圆 锥的 体积 为 ( )

- A. 2v3T 
- B. 3v37 
- C. 6v37 
- D. 9v3r

—x* —2ax- 
\\\#\\\#\\\# 0. 已知 函数 为一 (r) = VE \\$\begin\{vmatrix\} X , 在 R 上单 调递 增 , 则 a 取值 的范 围是 ( )
e +In\$\\left(x+1\\right)\\$,x 20

an

第 1 页 / 共 7 页




- A. \$\\left(-oo,0] 
- B. \\$\left[-L01

c. [-bl\right]\$


- D. \\$\left[0,+co\right)\$


\\\#\\\#\\\# 7. 当 xf [0,20\\right]\\$ Hf, 乡量羞y二sin〉〈与叉二2sin〔3〕〔一吾〕的交'「祟亢个数茭g〈 )


- A. 3 
- B. 4


- C. 6


- D. 8


\#\#\# 8. 已知 函数 为 D 的定 义域 为 R,(r) > 丫 \$\\left(x-D+ 丫 (x2\\right)\\$ , 且当 x<3 时个 (x) =x , 则下 列结 论中 一定 正确 的是 ( )

- A. f(10)>100

c. 十 L0) < 1000


- B. f(20)>1000


- D. f(20) <10000

二 、 选择 题 : 本题 共 3 小题 , 每小 题 6 分 , 共 18 分 . 在每 小题 给出 的选 项中 , 有多 项符 合题 目要 求 . 全部 选对 得 6 分 , 部分 选对 的得 部分 分 , 选对 但不 全的 得部 分分 , 有选 错的 得 0

分 .


\#\#\# 9. 为了 解推 动出 口后 的亩 收入 ( 单位 : 万元 ) 情况 , 从该 种植

均值 王 =
\#\#\# 2. 1, 样本 方差 s2 = 
\#\#\# 0. 01, 已知 该种 植

P\$\\left(Z <u+o\\right)\\$ x
\#\#\# 0. 8413 )

A P\$\\left(X>2\\right)\\$>02

c. P\$\\left(Y >2\\right)\\$>
\#\#\# 0. 5


\#\#\# 10. HMB f(x) =\$\\left(x-12 (x4\\right)\\$. I \$\\left(
AL x =39E S09 的极 小值 点


- C. 当 1<x<2 时 ,-4< 丨 (2x-UD < 0


\\\#\\\#\\\# 11. 造型 j 可以 做成 美丽 的丝 带 , 将其 看作 图

\\right)\\$

〕

出口 后的 畜收 入乙 服从 正态 分布 N( 二 5 ) , 则 (

B

D

)

区抽 取样 本 , 得到 推动 出口 后亩 收入 的样 本区 以往 的亩 收入 又服 从正 态分 布 N (
\#\#\# 1. 8,
\#\#\# 0. 1 ) , 假设 推动

若随 机变 量 z 服从 正态 分布 N(xo2) ,

. P\$\\left(X>2\\right)\\$ <
\#\#\# 0. 5

、 P\$\\left(Y>2\\right)\\$<
\#\#\# 08. 当 0 <

x<im, f(x)< f(x’)

. 4-1<x<0, f\$\\left(2-x\\right)\\$> f(x)

线 C 的一 部分 . 已知 C 过坐 标原 点 
\#\#\# 0. 且 C 上的 点满 足横 坐标 大于 -2 Ble (2,0) 的距 离与 到定 直线 x = 4\$\\left(4 < 0\\right)\\$ 的距 离之 积为 4, 则 〔 )

第 2 页 / 共 7 页


YA


- A. a=- 
- B. 点 (2V2,0) 在 C 上


- C. C 在第 一象 限的 点的 纵坐 标的 最大 值为 1 
- D. 当点 (xm,) 在 C 上时 , 加 <

加十 2

三 、 填空 题 : 本题 共 3 小题 , 每小 题 5 分 , 共 15 分 .

\#\#\# 12. 设双 公线 C: 一友 =1Ca > 0,b>0) 的左 右焦 点分 别为 之 、 及 , 过友 作平 行于 辆的 直线 交 C 于 4, 卯两 点 , 若  \end\{vmatrix\}\$ 4 厂 13,| 48 厂 10, 则 C 的离 心率 为


\\\#\\\#\\\# 13. 若曲 线 y=e“ + x 在点 (0,1) 处的 切线 也是 曲线 = In\\$\left(x + +a 的切 线 , 则 4 =


\#\#\# 14. 甲 、 乙两 人各 有四 张卡 片 , 每张 卡片 上标 有一 个数 字 , 甲的 卡片 上分 别标 有数 字 1,3,5,7, 乙的 卡片 上分 别标 有数 字 2,4,6,8, 两人 进行 四轮 比赛 , 在每 轮比 赛中 , 两人 各自 从自 己持 有的 卡片 中随 机选
一张 , 并比 较所 选卡 片上 数字 的大 小 , 数字 大的 人得 1 分 , 数字 小的 人得 0 分 , 然后 各自 弃置 此轮 所选 的卡 片 ( 弃置 的卡 片在 此后 的轮 次中 不能 使用 〉 . 则四 轮比 赛后 , 甲的 总得 分不 小于 2 的概 率为 ,
四 、 解答 题 : 本题 共 5 小题 , 共 77 分 . 解答 应写 出文 字说 明 、 证明 过程 或演 算步 骤 .


\#\#\# 15. 记 A4BC 内角 小 B、C 的对 边分 别为 a,5,c, 已知 sinC= vV2cosB,a2+D2 \_c2 =V2ab

C1\right)\$ RB;

(2) 若 a4BC 的面 积为 3+ v 引 , 求 .

2
+ 公 =1\\$\left(a >5 >0\right)\$ 上两 点 .
b

8] \\\&


\\\#\\\#\\\# 16. 已知 4(0,3) 和尸〔3皇为 椭圆 C:

C1) 求 C 的离 心率 ;
(2) 若过 P 的直 线 / 交 C 于另 一点 8, 且 A4BP 的面 积为 9, 求 1 的方 程 .


\\\#\\\#\\\# 17. 如图 , 四楂 锥 P- 4BCD 中 ,P4 L 底面 48CD,P4= 4C= 2 ,BC = 4B =v
\\\#\\\#\\\# 3. 第 3 页 / 共 7 页



(1) 若 4D LPB, iE: AD// 平面 PBC ;

(2) 若 4D L DC , 且二 面


\\\#\\\#\\\# 18. 已知 函数 广 r) = In

V42

x 4
2-x

A-—CP D iwiesxttiy 求 4
- D. 7

ax + b(x—1)°

(1) \\\#b=0, Hf'(x)20, Raita;

(2) 证明 : 曲线 7= 八 (

(3) 若个 (r) > -2 当且 仅当 1<x<2 , 求 》 的取 值范

X) 是中 心对 称图 形 ;

E


\\\#\\\#\\\# 19. 设力 为正 整数 , 数列 Q,,a,...,Qa, 是公 差不 为 0 的等 差数 列 , 若从 中删 去两 项 a 和 Q \\$\left(i < 门后 剩余

的 4 项可 被平 均分 为加 组 , 日每 组的 4 个数 都能 构成 等差 数列 , 则称 数列 Q,ao,...,Qy 是 ( \right)\$ 一可 分数 列 .

CD) 写出 所有 的 \\$\left( 月 ,1Ei < fS6, (BIA, a\right)\$,....0, 是 ( 门一 可分 数列 ;

(2) 当厂 >3 时 , 证明 : 数列 Q,do,...,Qy 是 (2,13) 一可 分数 列 ;

(3) 从 1 2,...,477 十 2 4

习 , 证明 : 砀>羞

P 一次 任取 两个 数 i 和 i < 门 , 记数 列 Q,,ao,...,Qa2 是 \\$\left( 门一 可分 数列 的概 率为

第 4 页 / 共 7 页绝 密文 启用 前
2024 年普 通高 等学 校招 生全 国统 一考 试 ( 新课 标 I 卷 》

数学
本试 卷共 10 页 ,19 小题 , 满分 150 分 .
注意 事项 :

\#\#\# 1. 答题 前 , 先将 自己 的姓 名 、 准考 证号 、 考场 号 、 座位 号填 写在 试卷 和答 题卡 上 , 并将 准考 证号 条形 码粘 贴在 答题 卡上 的指 定位 置 .

\#\#\# 2. 选择 题的 作答 : 每小 题选 出答 案后 , 用 28 铅笔 把答 题卡 上对 应题 目的 答案 标号 涂黑 . 写在 试卷 、 草稿 纸和 答题 卡上 的非 答题 区域 均无 效 .

\#\#\# 3. 填空 题和 解答 题的 作答 : 用黑 色签 字笔 直接 答在 答题 卡上 对应 的答 题区 域内 . 写在 试卷 、 草稿 纸和 答题 卡上 的非 答题 区域 均无 效 .

\#\#\# 4. 考试 结束 后 , 请将 本试 卷和 答题 卡一 并上 交 .

一 、 选择 题 ; 本题 共 8 小题 , 每小 题 5 分 , 共 40 分 . 在每 小题 给出 的四 个选 项中 , 只有
一个 选项 是正 确的 , 请把 正确 的选 项填 涂在 答题 卡相 应的 位置 上 .
【1 题答 案 】

【 答案 】A

【2 题答 案 】

【 答案 】C

【3 题答 案 】

【 答案 】D

【4 题答 案 】

【 答案 】A

【5 题答 案 】

【 答案 】 B

【6 题答 案 】

【 答案 】 B

【7 题答 案 】

【 答案 】C

第 5 页 / 共 7 页



【8 题答 案 】
【 答案 】B
二 、 选择 题 : 本题 共 3 小题 , 每小 题 6 分 , 共 18 分 . 在每 小题 给出 的选 项中 , 有多 项符 合题 目要 求 . 全部 选对 得 6 分 , 部分 选对 的得 部分 分 , 选对 但不 全的 得部 分分 , 有选 错的 得 0
分 .
(9 题答 案 】
【 答案 】BC
【10 题答 案 】
【 答案 】ACD
(11 题答 案 】
【 答案 】ABD
三 、 填空 题 : 本题 共 3 小题 , 每小 题 5 分 , 共 15 分 .
【12 题答 案 】

, 3
【 答案 】 井

2

(13 题答 案 】
【 答案 】In2
【14 题答 案 】
【 答案 ] 一梁 
\#\#\# 0. 5
四 、 解答 题 : 本题 共 5 小题 , 共 77 分 . 解答 应写 出文 字说 明 、 证明 过程 或演 算步 骤 .
【15 题答 案 】
(S41\right)\$ 8 = 与
(2) 2v2
【16 题答 案 】

【 答案 】CD 二

(2) 直线 1 的方 程为 3x\\\~2y 一 6=0 或 x\\\~2y =
\\\#\\\#\\\# 0. 【17 题答 案 】
【 答案 】(1〉 证明 见解 析

(2) 3

第 6 页 / 共 7 页



【18 GAR)

【 答案 】C1〉 -2

2
(2) 证明 见解 析 (3) b2--

【19 题答 案 】

(\\\#1 (1) (12),(1,6),(
\\\#\\\#\\\# 5. 6)

(2) 证明 见解 析 (3 证明 见解 析第 7 页 / 共 7 页





\vfill
\begin{center}
    \small{本文档由数学试卷转换器自动生成}
\end{center}

\end{document} 